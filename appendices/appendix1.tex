\chapter{Delves Coordinates}\label{delves}
This section contains the detailed derivation of the three-body Schr{\"o}dinger equation written in Delves coordinates. The coordinate definitions are the same as in \cref{delvescoord}. Here we simply repeat some previously stated definitions for a clearer overview. The Jacobi vectors in each coordinate set are defined by

\begin{subequations}
	\begin{align}
	r_{k} &= \rho \sin{\alpha_{k}},\\
	R_{k} &= \rho \cos{\alpha_{k}}
	\end{align}
\end{subequations}
and the Delves hyperangle $\alpha_{k}$ is defined by

\begin{equation}
\alpha_{k} = \arctan\bigg(\frac{r_{k}}{R_{k}}\bigg), \quad 0\leq \alpha_{k} \leq \frac{\pi}{2}.
\end{equation}
The other angle in this coordinate set is the angle between the two vectors $\mathbf{r}_{k}$ and $\mathbf{R}_{k}$, which is given by

\begin{equation}
\cos{\theta_{k}} = \frac{\mathbf{r}_{k} \cdot \mathbf{R}_{k}}{r_{k} R_{k}}, \quad  0\leq \theta \leq \pi.
\end{equation}

We choose one arbitary set to work in and suppress the indices from hereon. The mass weighted Schr{\"o}dinger equation for the stationary wavefunction $\Psi$ of a three-body system -- with position vectors $\mathbf{x}_i$ and masses $m_i$, ($i=1,2,3$) -- interacting pairwise through a potential $V$, is given by

\begin{equation}
-\frac{1}{2} \sum_{i=1}^{3} m^{-1}_{i} \nabla^{2}_i \Psi + V\Psi = E \Psi. 
\end{equation}
Where $\nabla_i^{2}$ is the Laplace operator for particle $i$, which in spherical coordinates reads

\begin{equation}
\nabla_i^2 = \frac{1}{r_i^{2}}\frac{\partial}{\partial r_i} \bigg(r_i^{2} \frac{\partial}{\partial r_i}\bigg) - \frac{L_i^{2}}{r_i^{2}},
\end{equation}
in which $L_i$ is the angular momentum operator associated with the vector $\mathbf{r}_i$. 

The kinetic energy for three particles in the mass normalized Jacobi coordinates was given in \Cref{eq:6}. Thus, after separating out the center-of-mass coordinate, the Schr{\"o}dinger equation for the internal motion is simply 

\begin{equation}
-\frac{1}{2\mu} \bigg(\nabla^{2}_{\mathbf{r}} + \nabla^{2}_{\mathbf{R}}\bigg) \Psi + V\Psi = E \Psi. 
\end{equation}

For three identical particles, the squared orbital angular momentum operators associated with the Jacobi vectors are given by  

\begin{equation}
L^{2}_r = L^{2}_{R} = -\frac{1}{\sin{\theta}} \frac{\partial}{\partial{\theta}} \bigg( \sin{\theta} \frac{\partial}{\partial{\theta}} \bigg).
\end{equation}
Change of coordinates subsequently results in the following transformations for the partial derivatives of the vector $\mathbf{r}$

\begin{align}
	\frac{\partial}{\partial r}        &= \frac{\partial\alpha}{\partial r} \frac{\partial}{\partial\alpha} +  \frac{\partial\rho}{\partial r} \frac{\partial}{\partial\rho} = \frac{1}{\rho}\cos{\alpha}\frac{\partial}{\partial \alpha} + \sin{\alpha}\frac{\partial}{\partial \rho}, \\
	\frac{\partial^2}{\partial r^2} &= \bigg( \frac{1}{\rho} \cos\alpha \frac{\partial}{\partial\alpha} + \sin\alpha \frac{\partial}{\partial\rho}\bigg) \bigg( \frac{1}{\rho} \cos\alpha \frac{\partial}{\partial\alpha} + \sin\alpha \frac{\partial}{\partial\rho}\bigg) \nonumber \\
	&= \frac{1}{\rho^2} \cos^2\alpha \frac{\partial^2}{\partial\alpha^{2}} - \frac{1}{\rho^2} \sin(2\alpha) \frac{\partial}{\partial\alpha} + \sin^2\alpha \frac{\partial^2}{\partial\rho^{2}} \nonumber \\
	&+ \frac{1}{\rho} \cos^2\alpha \frac{\partial}{\partial\rho} + \frac{1}{\rho} \sin(2\alpha) \frac{\partial^2}{\partial\alpha \partial\rho}.
\end{align}
Similarly, the partial derivatives with respect to the vector $\mathbf{R}$ transform as

\begin{align}
	\frac{\partial}{\partial R}        &= \frac{\partial\alpha}{\partial R} \frac{\partial}{\partial\alpha} +  \frac{\partial\rho}{\partial R} \frac{\partial}{\partial\rho} = -\frac{1}{\rho}\sin{\alpha}\frac{\partial}{\partial \alpha} + \cos{\alpha}\frac{\partial}{\partial \rho},  \\
	\frac{\partial^2}{\partial R^2}&= \bigg( -\frac{1}{\rho} \sin\alpha \frac{\partial}{\partial\alpha} + \cos\alpha \frac{\partial}{\partial\rho}\bigg)  \bigg( -  \sin\alpha \frac{\partial}{\partial\alpha} + \cos\alpha \frac{\partial}{\partial\rho}\bigg) \nonumber \\
	&= \frac{1}{\rho^2} \sin^2\alpha\frac{\partial^2}{\partial\alpha^{2}} + \frac{1}{\rho^2} \sin(2\alpha)\frac{\partial}{\partial\alpha} + \cos^2\alpha \frac{\partial^2}{\partial \rho^2} \nonumber \\
	&+\frac{1}{\rho}\sin^2\alpha\frac{\partial}{\partial \rho}- \frac{1}{\rho} \sin(2\alpha) \frac{\partial^2}{\partial\alpha \partial\rho}.
\end{align}
Finally, the sum of the two Laplacian operators now reads 

\begin{align}
\nabla^2_{\mathbf{r}} + \nabla^2_{\mathbf{R}} &= \frac{2}{r}\frac{\partial}{\partial r} +  \frac{2}{R} \frac{\partial}{\partial R}  +\frac{\partial^2}{\partial r^{2}} + \frac{\partial^2}{\partial R^{2}} - \frac{L^{2}_{r}}{r^2} - \frac{L^{2}_{R}}{R^2} \nonumber \\
&= \frac{4}{\rho^2} \cot(2\alpha) \frac{\partial}{\partial\alpha} + \frac{5}{\rho} \frac{\partial}{\partial\rho} + \frac{1}{\rho^2} \frac{\partial^2}{\partial\alpha^2} + \frac{\partial^2}{\partial\rho^2} \nonumber \\
&+ \frac{4}{\rho^2 \sin^2(2\alpha)\sin(\theta)} \frac{\partial}{\partial\theta} \bigg( \sin(\theta) \frac{\partial}{\partial{\theta}} \bigg) \nonumber \\
&= \frac{1}{\rho^5}\frac{\partial}{\partial\rho} \bigg( \rho^5 \frac{\partial}{\partial\rho} \bigg) + \frac{1}{\rho^2 \sin^2(2\alpha)}  \bigg( \frac{\partial}{\partial\alpha} \sin^2(2\alpha) \frac{\partial}{\partial\alpha} + \frac{4}{\sin\theta} \frac{\partial}{\partial\theta} \bigg).
\end{align}
The original Hamiltonian operator written in Delves coordinates can thus be expressed as 

\begin{equation}
H_0 = T_{\rho} + T_{\alpha} + T_{\theta} + V(\rho,\Omega).
\end{equation}  
Anticipating a rescaling of the wave function for subsequent removal of first derivatives with respect to $\rho$ and $\alpha$ warrants us to write the kinetic energy operators in the original Hamiltonian in the following ways: Let the hyperradial kinetic energy operator $T_{\rho}$ be expressed as

\begin{equation}\label{eq:kinetic_rho}
\begin{aligned}
T_{\rho} &= -\frac{1}{2\mu} \Big[ \frac{1}{\rho^5}\frac{\partial}{\partial\rho} \Big( \rho^5 \frac{\partial}{\partial\rho} \Big)  \Big] \\ 
&= -\frac{1}{2\mu} \Big[ \rho^{-5/2} \Big( \rho^{5/2} \frac{5}{\rho} \frac{\partial}{\partial\rho} + \rho^{5/2} \frac{\partial^2}{\partial\rho^2} \Big) \rho^{-5/2} \rho^{5/2} \Big]\\
&= -\frac{1}{2\mu} \rho^{-5/2} \Big[  -\frac{15}{4} \frac{1}{\rho^2} + \frac{\partial^2}{\partial\rho^2} \Big] \rho^{5/2}
\end{aligned}
\end{equation}

and let the kinetic energy operators for the hyperangles -- that is, the kinetic energy operator for the Delves angle $T_{\alpha}$ and the kinetic energy operator for the angular momentum of the Jacobi vectors $T_{\theta}$ -- be expressed as 

\begin{equation}\label{eq:kinetic_alpha}
\begin{aligned}
T_{\alpha} &= -\frac{1}{2\mu}  \frac{1}{\rho^2 \sin^2(2\alpha)}  \bigg[ \frac{\partial}{\partial\alpha} \sin^2(2\alpha) \frac{\partial}{\partial\alpha} \bigg]\\ 
&= -\frac{1}{2\mu} \frac{1}{\rho^2} \bigg[ \frac{\partial^2}{\partial\alpha^2} + 4\cot(2\alpha) \frac{\partial}{\partial\alpha} \bigg]\\
&= -\frac{1}{2\mu} \frac{1}{\rho^2} \bigg[ \sin^{-1}(2\alpha) \bigg(\sin(2\alpha)\frac{\partial^2}{\partial\alpha^2} + 4\cos(2\alpha) \frac{\partial}{\partial\alpha} \bigg) \sin^{-1}(2\alpha) \sin(2\alpha) \bigg]\\
&= -\frac{1}{2\mu} \frac{1}{\rho^2}\sin^{-1}(2\alpha) \bigg[ \frac{\partial^2}{\partial\alpha^2} + 4 \bigg] \sin(2\alpha),
\end{aligned} 
\end{equation}

and

\begin{align}                  
T_{\theta} &= -\frac{1}{2\mu} \bigg[ \frac{4}{\rho^2 \sin^2(2\alpha)\sin\theta} \frac{\partial}{\partial\theta} \bigg( \sin\theta \frac{\partial}{\partial\theta} \bigg) \bigg]\nonumber\\ 
&= -\frac{1}{2\mu} \bigg[ \frac{1}{\rho^2 \sin^2\alpha\cos^2\alpha\sin\theta} \frac{\partial}{\partial\theta} \bigg( \sin\theta \frac{\partial}{\partial\theta} \bigg) \bigg]
\end{align}
respectively. Removal of first derivatives with respect to $\rho$ and $\alpha$ is now possible by rescaling the total wave function such that $\Psi = \rho^{-5/2}(\sin(2\alpha))^{-1}\psi$. The corresponding transformation of the Hamiltonian is then

\begin{align}
H&= \rho^{5/2}\sin(2\alpha) H_0 \rho^{-5/2}(\sin(2\alpha))^{-1}\nonumber\\
&= -\frac{1}{2\mu} \bigg[ \frac{\partial^2}{\partial\rho^2} - \frac{15}{4\rho^2} + \frac{1}{\rho^2}\bigg( \frac{\partial^2}{\partial\alpha^2} + 4 + \frac{1}{\sin^2\alpha\cos^2\alpha\sin\theta} \frac{\partial}{\partial\theta} \bigg( \sin\theta \frac{\partial}{\partial\theta} \bigg) \bigg) \bigg]\nonumber\\
&= -\frac{1}{2\mu} \bigg[ \frac{\partial^2}{\partial\rho^2} + \frac{1}{\rho^2}\bigg( \frac{\partial^2}{\partial\alpha^2} + \frac{1}{\sin^2\alpha\cos^2\alpha\sin\theta} \frac{\partial}{\partial\theta} \bigg( \sin\theta \frac{\partial}{\partial\theta} \bigg) \bigg) + \frac{1}{4\rho^2} \bigg]\nonumber\\
&= -\frac{1}{2\mu}\frac{\partial^2}{\partial\rho^2} + \frac{\Lambda^2 - 1/4}{2\mu\rho^2},
\end{align}   
where $\Lambda^2$ contains all the hyperangular kinetic energy variables. The squared grand angular momentum is in this case given by 

\begin{equation}
\Lambda^2 = -\frac{\partial^2}{\partial\alpha^2} - \frac{1}{\sin^2\alpha\cos^2\alpha\sin\theta} \frac{\partial}{\partial\theta} \bigg( \sin\theta \frac{\partial}{\partial\theta}\bigg).
\end{equation}
The Schr{\"o}dinger equation can thus be written

\begin{equation}
\bigg(-\frac{1}{2\mu}\frac{\partial^2}{\partial\rho^2} + \frac{\Lambda^2 - 1/4}{2\mu\rho^2} + V(\rho,\alpha,\theta)\bigg) \psi(\rho,\alpha,\theta) = E \psi(\rho,\alpha,\theta),
\end{equation}
where $E$ is the internal energy. The corresponding volume element is proportional to $\rho^5\sin^2\alpha\cos^2\alpha\sin\theta\, d\rho\, d\alpha\, d\theta$. Since the rescaled wavefunction needs to be square-integrable for a bound state, the boundary conditions are given by

\begin{align}
	\psi(0,\alpha,\theta) &= 0,\\
	\psi(\rho,0,\theta)    &= \psi(\rho,\frac{\pi}{2},\theta) = 0,\\
	\frac{\partial\psi}{\partial\theta}\bigg\rvert_{\theta = 0} &= \frac{\partial\psi}{\partial\theta}\bigg\rvert_{\theta = \pi} = 0.
\end{align}   