\documentclass{article}

\usepackage{amsmath}
\usepackage{mathtools}
\usepackage{amsfonts}
\usepackage{amssymb}
\usepackage{biblatex}
\usepackage{multirow}
\usepackage{graphicx}
\usepackage{caption}
\usepackage{bm}
\usepackage{gensymb}
\usepackage{siunitx}
\usepackage{float}


\title{Efimov}
\author{Kajsa-My Blomdahl}
\date{April 2018}
\begin{document}

\maketitle

\section{The Hyperspherical Method}

\subsection{Coordinate Transformations}
\subsubsection{Delves Coordinates}

Blabla labframe coordinates

\begin{subequations} \label{eq:1}
Jacobi vectors:
\begin{align}
        \vec{x} 		&= \vec{r}_2 - \vec{r}_1\\
        \vec{y} 		&= \vec{r}_3 - \frac{1}{2}(\vec{r}_1 + \vec{r}_2)\\
        \vec{X}_{cm} 	&= \frac{1}{3} (\vec{r}_1 + \vec{r}_2 + \vec{r}_3)
\end{align}
\end{subequations}

Exchange of particles
\begin{equation}
\hat{P}_{13} = \left \{ \begin{aligned}
        \vec{x} \, ' &= \vec{r}_2 - \vec{r}_3 &&= \frac{1}{2}\vec{x} - \vec{y} \\
        \vec{y} \, ' &= \vec{r}_1 - \frac{1}{2}(\vec{r}_3 + \vec{r}_2) &&= -\frac{1}{2} (\frac{3}{2} \vec{x} + \vec{y})
       \end{aligned}
 \right.
 \end{equation}
 
\begin{equation}
\hat{P}_{23} = \left \{ \begin{aligned}
        \vec{x} \, '' &= \vec{r}_3 - \vec{r}_1 &&= \frac{1}{2}\vec{x} + \vec{y} \\
        \vec{y} \, '' &= \vec{r}_2 - \frac{1}{2}(\vec{r}_1 + \vec{r}_3) &&= -\frac{1}{2} (\frac{3}{2} \vec{x} - \vec{y})
       \end{aligned}
 \right.
 \end{equation}

\begin{subequations}
Introducing hyperspherical coordinates:
\begin{align}
        x &= \sqrt{2} \rho \sin(\alpha)\\
        y &= \sqrt{\frac{3}{2}} \rho \cos(\alpha)\\
\end{align}
\end{subequations}

Hyperspherical coordinates:
\[\arraycolsep=1.4pt\def\arraystretch{2.2}
     \left. \begin{array}{lr}
        \rho = \displaystyle \Big( \frac{1}{2}x^2 + \frac{2}{3}y^2 \Big) ^{1/2} &,  0\leq \rho < \infty\\
        \tan{\alpha} = \displaystyle \frac{\sqrt{3}}{2} \frac{x}{y} &,  0\leq \alpha < \frac{\pi}{2}\\
        \cos{\theta} = \displaystyle \frac{\vec{x} \cdot \vec{y}}{xy} &,  0\leq \theta < \pi
        \end{array}\right.
  \]

\begin{subequations}
Jacobi vectors:
\begin{align}
        x' &= \Big(\frac{1}{4} x^2 +y^2 -\vec{x} \cdot \vec{y}\Big)^{1/2} = \frac{\rho}{\sqrt{2}} \big( \sin^2(\alpha) + 3\cos^2(\alpha) - \sqrt{3}\sin(2\alpha)\cos{\theta}\big)^{1/2}\\
        x'' &= \Big(\frac{1}{4} x^2 +y^2 +\vec{x} \cdot \vec{y}\Big)^{1/2} = \frac{\rho}{\sqrt{2}} \big( \sin^2(\alpha) + 3\cos^2(\alpha) + \sqrt{3}\sin(2\alpha)\cos{\theta}\big)^{1/2}
\end{align}
\end{subequations}

Volume element from the transformation is $dr_1dr_2dr_3=3/2dxdydX_{cm}$.\\


(The massweighted Schr{\"o}dinger equation of a N-body system with position vectors $\mathbf{r}_k$ and masses $m_k$, ($k=1,...,N$), is given by)

\begin{equation}
\Bigg(-\frac{\hbar^2}{2} \sum_{k=1}^{N} m^{-1}_{k} \nabla^{2}_{\mathbf{r}_{k}} \Psi + V\Psi = E \Psi \Bigg)
\end{equation}

where the Laplacian is

\begin{equation}
\nabla^{2} = \Big( \frac{1}{r^2}\partial_{r} \big(r^2 \partial_{r}\big) - \frac{1}{\sin(\theta)} \partial_{\theta} \big( \sin(\theta) \partial_{\theta} \big) + \frac{1}{\sin^2{\theta}} \partial^{2}_{\phi} \Big) = \Big( \frac{1}{r^2} \partial_{r} \big( r^2 \partial_{r} \big) + \frac{L^2}{\hbar^2} \Big)
\end{equation}

The kinetic energy for three particles with identical masses is given by

\begin{equation}
\hat{T} = -\frac{\hbar^2}{2m}( \nabla^{2}_{r_1} + \nabla^{2}_{r_2} + \nabla^{2}_{r_3} )
\end{equation}

in hyperspherical coordinates this becomes 

\begin{equation}
\hat{T} = -\frac{\hbar^2}{2m}( 2\nabla^{2}_{x} + \frac{3}{2}\nabla^{2}_{y} + \frac{1}{3}\nabla^{2}_{X_{cm}} ),
\end{equation}

where

\begin{subequations}
\begin{align}
	\nabla^2_{x} &= \frac{1}{x^2}\frac{\partial}{\partial x} \Big( x^2 \frac{\partial}{\partial x} \Big) - \frac{\hat{l}^2_{x}}{x^2} = \frac{2}{x}\frac{\partial}{\partial x} + \frac{\partial^2}{\partial x^{2}} - \frac{\hat{l}^{2}_{x}}{x^2}\\
	\nabla^2_{y} &= \frac{1}{y^2}\frac{\partial}{\partial y} \Big( y^2 \frac{\partial}{\partial y} \Big) - \frac{\hat{l}^2_{y}}{y^2} = \frac{2}{y}\frac{\partial}{\partial y} + \frac{\partial^2}{\partial y^{2}} - \frac{\hat{l}^{2}_{y}}{y^2}
\end{align}
\end{subequations}

If spin interactions are excluded the total orbital angular momentum is zero and we have 

\begin{subequations}
\begin{align}
\hat{l}^{2}_{x} = \hat{l}^{2}_{y} = -\frac{1}{\sin(\theta)} \frac{\partial}{\partial{\theta}} \Big( \sin(\theta) \frac{\partial}{\partial{\theta}} \Big)
\end{align}
\end{subequations}



\begin{subequations}
\begin{align*}
        \frac{\partial \alpha}{\partial x} &= \frac{1}{\sqrt{2} \rho} \cos(\alpha), \quad \frac{\partial \rho}{\partial x} = \frac{1}{\sqrt{2}} \sin(\alpha) \\
        \frac{\partial \alpha}{\partial y} &= -\frac{\sqrt{6}}{3 \rho} \sin(\alpha), \quad \frac{\partial \rho}{\partial y} = \sqrt{\frac{2}{3}} \cos(\alpha)
\end{align*}
\end{subequations}

\begin{subequations}
\begin{align*}
        \frac{\partial}{\partial x}        &= \frac{\partial\alpha}{\partial x} \frac{\partial}{\partial\alpha} +  \frac{\partial\rho}{\partial x} \frac{\partial}{\partial\rho} \\
        \frac{\partial^2}{\partial x^2} &= \frac{1}{2} \Big( \frac{1}{\rho} \cos(\alpha) \frac{\partial}{\partial\alpha} + \sin(\alpha) \frac{\partial}{\partial\rho}\Big) \Big( \frac{1}{\rho} \cos(\alpha) \frac{\partial}{\partial\alpha} + \sin(\alpha) \frac{\partial}{\partial\rho}\Big) \\
                                                     &= \frac{1}{2} \Big[ \frac{1}{\rho^2} \cos(\alpha) \frac{\partial}{\partial\alpha} \Big( \cos(\alpha) \frac{\partial}{\partial\alpha} \Big) + \frac{1}{\rho} \cos(\alpha) \frac{\partial}{\partial\alpha} \Big( \sin(\alpha) \frac{\partial}{\partial\rho} \Big) + \sin(\alpha) \frac{\partial}{\partial\rho} \Big( \frac{1}{\rho} \cos(\alpha) \frac{\partial}{\partial\alpha} + \sin(\alpha) \frac{\partial}{\partial\rho} \Big)   \Big] \\
                                                     &= \frac{1}{2} \Big[ -\frac{2}{\rho^2} \cos(\alpha) \sin(\alpha) \frac{\partial}{\partial\alpha} + \frac{1}{\rho^2} \cos^2(\alpha) \frac{\partial^2}{\partial\alpha^{2}} + \frac{1}{\rho} \cos^2(\alpha) \frac{\partial}{\partial\rho} + \frac{2}{\rho} \cos(\alpha)\sin(\alpha) \frac{\partial^2}{\partial\alpha \partial\rho} + \sin^2(\alpha) \frac{\partial^2}{\partial\rho^{2}}\Big] \\
                                                     &= \frac{1}{2} \Big[ \frac{1}{\rho^2} \cos^2(\alpha) \frac{\partial^2}{\partial\alpha^{2}} - \frac{1}{\rho^2} \sin(2\alpha) \frac{\partial}{\partial\alpha} + \sin^2(\alpha) \frac{\partial^2}{\partial\rho^{2}} + \frac{1}{\rho} \cos^2(\alpha) \frac{\partial}{\partial\rho} + \frac{1}{\rho} \sin(2\alpha) \frac{\partial^2}{\partial\alpha \partial\rho}\Big]
\end{align*}
\end{subequations}

\begin{subequations}
\begin{align*}
        \frac{\partial}{\partial y}        &= \frac{\partial\alpha}{\partial y} \frac{\partial}{\partial\alpha} +  \frac{\partial\rho}{\partial y} \frac{\partial}{\partial\rho} \\
        \frac{\partial^2}{\partial y^2}&= \Big( -\frac{\sqrt{6}}{3\rho} \sin(\alpha) \frac{\partial}{\partial\alpha} + \sqrt{\frac{2}{3}} \cos(\alpha) \frac{\partial}{\partial\rho}\Big)  \Big( -\frac{\sqrt{6}}{3}  \sin(\alpha) \frac{\partial}{\partial\alpha} + \sqrt{\frac{2}{3}} \cos(\alpha) \frac{\partial}{\partial\rho}\Big) \\
                                                    &= \frac{2}{3} \Big[ \frac{1}{\rho^2} \sin(\alpha) \frac{\partial}{\partial\alpha} \Big( \sin(\alpha) \frac{\partial}{\partial\alpha}\Big) - \frac{1}{\rho} \sin(\alpha) \frac{\partial}{\partial\alpha} \Big( \cos(\alpha) \frac{\partial}{\partial\rho} \Big) - \cos(\alpha) \frac{\partial}{\partial\rho} \Big( \frac{1}{\rho} \sin(\alpha) \frac{\partial}{\partial\alpha} \Big) + \cos^2(\alpha)\frac{\partial^2}{\partial\rho^{2}} \Big] \\
                                                    &= \frac{2}{3} \Big[ \frac{2}{\rho^2} \sin(\alpha) \cos(\alpha) \frac{\partial}{\partial\alpha} + \frac{1}{\rho^2} \sin^2(\alpha)\frac{\partial^2}{\partial\alpha^{2}} + \frac{1}{\rho} \sin^2(\alpha) \frac{\partial}{\partial\rho} - \frac{2}{\rho}\sin(\alpha)\cos(\alpha) \frac{\partial^2}{\partial\alpha \partial\rho} +\cos^2(\alpha)\frac{\partial^2}{\partial\rho^{2}}  \Big]\\
                                                    &= \frac{2}{3} \Big[ \frac{1}{\rho^2} \sin^2(\alpha)\frac{\partial^2}{\partial\alpha^{2}} + \frac{1}{\rho^2} \sin(2\alpha)\frac{\partial}{\partial\alpha} + \cos^2(\alpha) \frac{\partial^2}{\rho^2} - \frac{1}{\rho} \sin(2\alpha) \frac{\partial^2}{\partial\alpha \partial\rho} \Big]
\end{align*}
\end{subequations}

\begin{subequations}
\begin{align*}
	2\nabla^{2}_{x} + \frac{3}{2}\nabla^{2}_{y} &= \frac{4}{x}\frac{\partial}{\partial x} +  \frac{3}{y} \frac{\partial}{\partial y}  +2\frac{\partial^2}{\partial x^{2}} + \frac{3}{2} \frac{\partial^2}{\partial y^{2}} - 2\frac{\hat{l}^{2}_{x}}{x^2} - \frac{3}{2}\frac{\hat{l}^{2}_{y}}{y^2}\\
									&= \frac{4}{\rho^2} \cot(2\alpha) \frac{\partial}{\partial\alpha} + \frac{5}{\rho} \frac{\partial}{\partial\rho} + \frac{1}{\rho^2} \frac{\partial^2}{\partial\alpha^2} + \frac{\partial^2}{\partial\rho^2} + \frac{4}{\rho^2 \sin^2(2\alpha)\sin(\theta)} \frac{\partial}{\partial\theta} \Big( \sin(\theta) \frac{\partial}{\partial{\theta}} \Big)\\
									&= \frac{1}{\rho^5}\frac{\partial}{\partial\rho} \Big( \rho^5 \frac{\partial}{\partial\rho} \Big) + \frac{1}{\rho^2 \sin^2(2\alpha)}  \Big( \frac{\partial}{\partial\alpha} \sin^2(2\alpha) \frac{\partial}{\partial\alpha} + \frac{4}{\sin(\theta)} \frac{\partial}{\partial\theta} \Big)
\end{align*}
\end{subequations}

The kinetic energy operators expressed in Delves hypersherical coordinates is thus

\begin{equation}
\hat{T} = \hat{T}_{\rho} +  \hat{T}_{\alpha}+\hat{T}_{\theta} 
\end{equation}

where

\begin{subequations}
\begin{align*}
\hat{T}_{\rho} &= -\frac{\hbar^2}{2m} \Big[ \frac{1}{\rho^5}\frac{\partial}{\partial\rho} \Big( \rho^5 \frac{\partial}{\partial\rho} \Big)  \Big]\\ 
                      &= -\frac{\hbar^2}{2m} \Big[ \rho^{-5/2} \Big( \rho^{5/2} \frac{5}{\rho} \frac{\partial}{\partial\rho} + \rho^{5/2} \frac{\partial^2}{\partial\rho^2} \Big) \rho^{-5/2} \rho^{5/2} \Big]\\
                      &= -\frac{\hbar^2}{2m} \rho^{-5/2} \Big[  -\frac{15}{4} \frac{1}{\rho^2} + \frac{\partial^2}{\partial\rho^2} \Big] \rho^{5/2}\\ \\
\hat{T}_{\alpha} &= -\frac{\hbar^2}{2m}  \frac{1}{\rho^2 \sin^2(2\alpha)}  \Big[ \frac{\partial}{\partial\alpha} \sin^2(2\alpha) \frac{\partial}{\partial\alpha} \Big]\\ 
                      &= -\frac{\hbar^2}{2m} \frac{1}{\rho^2} \Big[ \frac{\partial^2}{\partial\alpha^2} + 4\cot(2\alpha) \frac{\partial}{\partial\alpha} \Big]\\
                      &= -\frac{\hbar^2}{2m} \frac{1}{\rho^2} \Big[ \sin^{-1}(2\alpha) \Big(\sin(2\alpha)\frac{\partial^2}{\partial\alpha^2} + 4\cos(2\alpha) \frac{\partial}{\partial\alpha} \Big) \sin^{-1}(2\alpha) \sin(2\alpha) \Big]\\
                      &= -\frac{\hbar^2}{2m} \frac{1}{\rho^2}\sin^{-1}(2\alpha) \Big[ \frac{\partial^2}{\partial\alpha^2} + 4 \Big] \sin(2\alpha)\\
                      &= -\frac{\hbar^2}{2m} \frac{1}{\rho^2}(\sin(\alpha)\cos(\alpha))^{-1} \Big[ \frac{\partial^2}{\partial\alpha^2} + 4 \Big] \sin(\alpha)\cos(\alpha)\\ \\
\hat{T}_{\theta} &= -\frac{\hbar^2}{2m} \Big[ \frac{4}{\rho^2 \sin^2(2\alpha)\sin(\theta)} \frac{\partial}{\partial\theta} \Big( \sin(\theta) \frac{\partial}{\partial\theta} \Big) \Big]\\ 
                      	&= -\frac{\hbar^2}{2m} \Big[ \frac{1}{\rho^2 \sin^2(\alpha)\cos^2(\alpha)\sin(\theta)} \frac{\partial}{\partial\theta} \Big( \sin(\theta) \frac{\partial}{\partial\theta} \Big) \Big]\\ 
\end{align*}                      		
\end{subequations} 

And we get the Hamiltonian 

\begin{equation}
H_0 = \hat{T}_{\rho} +  \hat{T}_{\alpha}+\hat{T}_{\theta} + V(\rho,\alpha,\theta)
\end{equation}

From now on $\hbar = 1$. To remove first derivatives with respect to $\rho$ and $\alpha$ we make the following transformation $\Psi = \rho^{-5/2}(\sin(\alpha)\cos(\alpha))^{-1}\psi$. This corresponds to the following transformation of the Hamiltonian 

\begin{subequations}
\begin{align*}
	H &= \rho^{5/2}\sin(\alpha)\cos(\alpha) H_0 \rho^{-5/2}(\sin(\alpha)\cos(\alpha))^{-1}\\
	    &= -\frac{1}{2m} \Big[ \frac{\partial^2}{\partial\rho^2} - \frac{15}{4\rho^2} + \frac{1}{\rho^2}\Big( \frac{\partial^2}{\partial\alpha^2} + 4 + \frac{1}{\sin^2(\alpha)\cos^2(\alpha)\sin(\theta)} \frac{\partial}{\partial\theta} \Big( \sin(\theta) \frac{\partial}{\partial\theta} \Big) \Big) \Big]\\
	    &= -\frac{1}{2m} \Big[ \frac{\partial^2}{\partial\rho^2} + \frac{1}{\rho^2}\Big( \frac{\partial^2}{\partial\alpha^2} + \frac{1}{\sin^2(\alpha)\cos^2(\alpha)\sin(\theta)} \frac{\partial}{\partial\theta} \Big( \sin(\theta) \frac{\partial}{\partial\theta} \Big) \Big) + \frac{1}{4\rho^2} \Big]\\
	    &= -\frac{1}{2m}\frac{\partial^2}{\partial\rho^2} + \frac{\Lambda^2 - 1/4}{2m\rho^2}
\end{align*}   
\end{subequations}

Where the Grand angular momentum operator is

\begin{equation}
\Lambda^2 = -\frac{\partial^2}{\partial\alpha^2} - \frac{1}{\sin^2(\alpha)\cos^2(\alpha)\sin(\theta)} \frac{\partial}{\partial\theta} \Big( \sin(\theta) \frac{\partial}{\partial\theta}\Big)
\end{equation}

The volume element is proportional to $\rho^5\sin^2(\alpha)\cos^2(\alpha)\sin(\theta)d\rho d\alpha d\theta$. Boundary conditions:
The wavefunction needs to be square-integrable, so $\Psi = 0$ at $\rho=0$ and $\alpha = 0 $ or $\pi$. Further, 

\begin{subequations}
\begin{align*}
	\psi(0,\alpha,\theta) &= 0\\
	\psi(\rho,0,\theta)    &= \psi(\rho,\frac{\pi}{2},\theta) = 0\\
	\frac{\partial\psi}{\partial\theta}\bigg\rvert_{\theta = 0} &= \frac{\partial\psi}{\partial\theta}\bigg\rvert_{\theta = \pi} = 0
\end{align*}   
\end{subequations} 



\subsubsection{Modified Smith-Whitten Coordinates}
In this section, we show how the three-body system can be represented in a symmetric way. The derivation of the Hamiltonian for this representation is described using a modified set of Smith-Whitten (democratic) coordinates. 

For a system of three particles, let $\mathbf{r}^i$ and $m_i$ be the position vector and mass of the $i$th particle. The set of Jacobi coordinates for the system is given by

\begin{subequations}
	\begin{align}
	\mathbf{x}^k 		&= \mathbf{r}^{j}_0 - \mathbf{r}^{i}_0,\\
	\vec{y}_k 		&= \vec{r}_k - \frac{1}{2}(\vec{r}_i + \vec{r}_j),\\
	\vec{X}_{cm} 	&= \frac{1}{3} \sum_{i=1}^{3}\vec{r}_i.
	\end{align}
\end{subequations}

The center of mass coordinate separates from the equations of motion and we will not consider it further. We can define a new set of mass normalized Jacobi coordinates. If the total mass $M$, the three particle reduced mass $\mu_3$, and the normalizing constants $d_k (k=1,2,3)$ are defined as 

\begin{align}
M &= \sum_{i=1}^{3}m_i,\\
\mu_3^2 &= \prod_{i=1}^{3}m_i/M,\\
d_k^2 &= \frac{m_k}{\mu_3}\frac{(m_i+m_j)}{M}. 
\end{align}

The set of mass normalized Jacobi coordinates are defined by

\begin{align}
\mathbf{r}_k &= d^{-1}_k\mathbf{x}_k,\\
\mathbf{R}_k &= d_k\mathbf{y}_k.  
\end{align}
The masses of the particles define angles that are useful to describe permutations of the system. If we consider an even permutation $(ijk)$ of the set $(123)$, then the obtuse angle $\beta_{ij}$ has the properties

\begin{subequations}
	\begin{align}
		&\beta_{ij} = -\beta_{ji}, \quad \beta_{ii} = 0,\\
		&\tan\beta_{ij} = -m_k/\mu,\\
		&d_{i}d_{j} \sin\beta_{ij} = 1,\\
		&d_{i}d_{j} m_{k} \cos\beta_{ij} = -\mu,\\
		&\beta_{12}+\beta_{23}+\beta_{31} = 2\pi
	\end{align}
\end{subequations}
Orthogonal transformations within the coordinate set are then given by 

\begin{equation}
	\begin{pmatrix}
		\mathbf{r}_j\\
		\mathbf{R}_j
	\end{pmatrix}
	=
	\begin{pmatrix}
	\cos\beta_{ij} & \sin\beta_{ij}\\
	-\sin\beta_{ij} & \cos\beta_{ij}
	\end{pmatrix}
	\begin{pmatrix}
	\mathbf{r}_i\\
	\mathbf{R}_i
	\end{pmatrix}
\end{equation}   

From hereon we choose one set to work in and suppress all vector indices. In six dimensional space, the components of the two vectors $\mathbf{r}$ and $\mathbf{R}$ can be regarded as the Cartesian components of a point that space. The kinetic energy is then given by

\begin{equation}
\hat{T} = -\frac{1}{2m}\Big(\Delta_{\mathbf{r}}+\Delta_{\mathbf{R}}\Big).
\end{equation} 

At any instant, three particles form a plane in $\mathbb{R}^3$. If we consider this plane to be the x-y plane, and define the internal motion of the particles within this plane in terms of  hyperspherical coordinates, our coordinate system must rotate in this plane. That is, we use a body-fixed coordinate system $XYZ$, which rotates with respect to the space fixed axis $X'Y'Z'$. [Details](spatial rotation). The internal coordinates $\rho$, $\Theta$, and $\Phi$ determine the size and shape of the triangle formed by the three particle system. With the $z$-axis perpendicular to the plane, Smith and Whitten [ref] defined these as   

\begin{subequations}
\begin{align*}
	r_x &= \rho \cos(\Theta)\cos(\Phi),\\
	r_y &= -\rho \sin(\Theta)\sin(\Phi),\\
	r_z &= 0\\
	R_x &= \rho \cos(\Theta)\sin(\Phi),\\
	R_y &= \rho \sin(\Theta)\cos(\Phi),\\
	R_z &= 0.
\end{align*}   
\end{subequations}
The distance between the particles are given by

\begin{align}
	x_3 = d_3 \mid\mathbf{r}_{3}\mid &= \frac{\rho d_3}{2^{1/2}}\big[1+\cos(2\Theta)\cos(2\Phi_3)\big]^{1/2}\\
	x_1 = d_1 \mid\mathbf{r}_{1}\mid &= d_1 \big[\cos^2\beta_{31}\mathbf{r}^2_{3} + \sin^2\beta_{31}\mathbf{R}^2_3 + 2\sin\beta_{31}\cos\beta_{31}\mathbf{r}_3\cdot\mathbf{R}_3\big]^{1/2}\\ \notag
	&= \frac{d_1\rho}{2^{1/2}} \big[\cos^2\beta_{31}(1 + \cos(2\Theta)\cos(2\Phi_3))\\ \notag
	&+ \sin^2\beta_{31}(1 - \cos(2\Theta)\cos(2\Phi_3))\\ \notag
	&+ 2\sin\beta_{31}\cos\beta_{31}\cos(2\Theta)\sin(2\Phi_3)\big]^{1/2}\\ \notag
	&= \frac{d_1\rho}{2^{1/2}} \big[1 + \cos(2\Theta)\big(\cos(\Phi_3)\cos(2\beta_{31}) + \sin(2\Phi_3)\sin(2\beta_{31})\big)\big]^{1/2}\\ \notag
	&= \frac{d_1\rho}{2^{1/2}}\big[1 + \cos(2\Theta)\cos(2\Phi_3 - 2\beta_{31})\big]^{1/2}\\ \notag
	&= \frac{d_1\rho}{2^{1/2}}\big[1 + \cos(2\Theta)\cos(2\Phi_1)\big]^{1/2}\\ \notag
	x_2 = d_2 \mid\mathbf{r}_{2}\mid
	&= d_2 \big[\cos^2\beta_{23}\mathbf{r}^2_{3} + \sin^2\beta_{23}\mathbf{R}^2_3 - 2\sin\beta_{23}\cos\beta_{23}\mathbf{r}_3\cdot\mathbf{R}_3\big]^{1/2}\\ \notag
	&= \frac{d_2\rho}{2^{1/2}} \big[1 + \cos(2\Theta)\big(\cos(\Phi_3)\cos(2\beta_{23}) - \sin(2\Phi_3)\sin(2\beta_{23})\big)\big]^{1/2}\\ \notag
	&= \frac{d_1\rho}{2^{1/2}}\big[1 + \cos(2\Theta)\cos(2\Phi_3 + 2\beta_{23})\big]^{1/2}\\ \notag
	&= \frac{d_1\rho}{2^{1/2}}\big[1 + \cos(2\Theta)\cos(2\Phi_2)\big]^{1/2}
\end{align}
Thus, $\Phi_j = \Phi_i-\beta_{ij}$ and

\begin{equation}
	x_k = \frac{d_k\rho}{2^{1/2}}\big[1 + \cos(2\Theta)\cos(2\Phi_k)\big]^{1/2}.
\end{equation}
Now we choose $\Phi_3=\Phi$

\begin{align}
	x_3 &= \frac{\rho d_3}{2^{1/2}}\big[1+\cos(2\Theta)\cos(2\Phi)\big]^{1/2}\\
	x_1 &= \frac{d_1\rho}{2^{1/2}}\big[1 + \cos(2\Theta)\cos(2\Phi + \epsilon_1)\big]^{1/2}\\
	x_2 &= \frac{d_1\rho}{2^{1/2}}\big[1 + \cos(2\Theta)\cos(2\Phi + \epsilon_2)\big]^{1/2}
\end{align}
where

\begin{align}
	\epsilon_1 &= -2\tan^{-1}(-m_2/\mu)\\
	\epsilon_2 &= 2\tan^{-1}(-m_1/\mu)
\end{align}
Now for three identical particles we get

\begin{align}
x_3 &= \frac{\rho}{3^{1/4}}\big[1+\cos(2\Theta)\cos(2\Phi)\big]^{1/2}\\
x_1 &= \frac{\rho}{3^{1/4}}\big[1 + \cos(2\Theta)\cos(2\Phi - 4\pi/3)\big]^{1/2}\\
x_2 &= \frac{\rho}{3^{1/4}}\big[1 + \cos(2\Theta)\cos(2\Phi + 4\pi/3)\big]^{1/2}
\end{align}

Now, with $\phi_k = \pi/2-2\Phi_k$, we get $\phi_j=\phi_i+2\beta_{ij}$, where ($-7\pi/2 \leq \phi_k < \pi/2$). Now with $2\beta_{ij} = -2\eta_{ij}$ and

\begin{align}
	&\eta_{ij} = -\eta_{ji}, \quad \eta_{ii} = 0,\\
	&\tan\eta_{ij} = m_k/\mu,\\
	&\eta_{12}+\eta_{23}+\eta_{31} = \pi
\end{align} 
We get

\begin{equation}
x_k = \frac{d_k\rho}{2^{1/2}}\big[1 + \sin\theta\sin\phi_k\big]^{1/2}.
\end{equation}

\begin{align}
x_3 &= \frac{d_3\rho}{2^{1/2}}\big[1+\sin\theta\sin\phi\big]^{1/2}\\
x_1 &= \frac{d_1\rho}{2^{1/2}}\big[1 + \sin\theta\sin(\phi-\varphi_1)\big]^{1/2}\\
x_2 &= \frac{d_1\rho}{2^{1/2}}\big[1 + \sin\theta\sin(\phi + \varphi_2)\big]^{1/2}
\end{align}

\begin{align}
\varphi_1 &= 2\tan^{-1}(m_2/\mu)\\
\varphi_2 &= 2\tan^{-1}(m_1/\mu)
\end{align}
Now we redefine $\phi'_k = \phi_k+7\pi/2$, so that the range is $0 \leq \phi'_k < 4\pi$. Then $\sin\phi_k = \cos\phi'_k$. We finally get ($2\Phi = 4\pi - \phi'$)

\begin{align}
x_3 &= \frac{d_3\rho}{2^{1/2}}\big[1+\sin\theta\cos\phi'\big]^{1/2}\\
x_1 &= \frac{d_1\rho}{2^{1/2}}\big[1 + \sin\theta\cos(\phi'-\varphi_1)\big]^{1/2}\\
x_2 &= \frac{d_1\rho}{2^{1/2}}\big[1 + \sin\theta\cos(\phi' + \varphi_2)\big]^{1/2}.
\end{align}
This is the same expression as Blume and Wang get, however the define there angles slightly different. Our interval is two times Blumes interval 

The area of the triangle formed by the three particles is the length of the vector given by

\begin{equation}
	\mathbf{A}=\frac{1}{2} (\mathbf{r}\times \mathbf{R})
\end{equation}

\begin{align}
&A =\frac{1}{2} (r_x R_y - r_y R_x) = \frac{1}{4}\rho^2\sin(2\Theta)\\
&\sin2\Theta = 4A/\rho^2
\end{align}
Since both the area and the hyperradius are positive quantities the angle must be in the range, $0\leq \Theta \leq \pi/4$. For some reason $0\leq \Phi < 2\pi$. The transformed coordinates then have the range $0\leq \theta \leq \pi/2$ and 

To describe how rotations of the BF system affects the derivatives in the SF system, introduce the infinitesimal rotations $\mathbf{\omega}$. The velocities of the SF system vectors are given by    

\begin{subequations}
	\begin{align*}
	 \dot{\mathbf{r}}' = \dot{\mathbf{r}} + \mathbf{\omega} \times \mathbf{r}\\
	 \dot{\mathbf{R}}' = \dot{\mathbf{R}} + \mathbf{\omega} \times \mathbf{R}\\
	\end{align*}   
\end{subequations} 

which is given explicitly by

\begin{align}
\begin{pmatrix}
       \dot{r}'_x \\
       \dot{r}'_y \\
       \dot{r}'_z \\
       \dot{R}'_x \\
       \dot{R}'_y \\
       \dot{R}'_z\\
 \end{pmatrix} 
 &= 
 \begin{pmatrix*}[c]
       \partial_{\rho} r_x & \partial_{\Theta} r_x & \partial_{\Theta} r_x & 0 & r_z & -r_y\\
       \partial_{\rho} r_y & \partial_{\Theta} r_y & \partial_{\Theta} r_y & -r_z & 0 & r_x\\
       \partial_{\rho} r_z & \partial_{\Theta} r_z & \partial_{\Theta} r_z & r_y & -r_x & 0\\
       \partial_{\rho} R_x & \partial_{\Theta} R_x & \partial_{\Theta} R_x & 0 & R_z & -R_y\\
       \partial_{\rho} R_y & \partial_{\Theta} R_y & \partial_{\Theta} R_y & -R_z & 0 & R_x\\
       \partial_{\rho} R_z & \partial_{\Theta} R_z & \partial_{\Theta} R_z & R_y & -R_x & 0\\
     \end{pmatrix*}
     \begin{pmatrix}
     \dot{\rho}\\
     \dot{\Theta}\\
     \dot{\Phi}\\
     \omega_x\\
     \omega_y\\
     \omega_z\\
     \end{pmatrix} \notag \\
     &=
     \begin{pmatrix*}[c]
       cc & -sc & -cs & 0 & 0 & ss\\
       -ss & -cs & -sc & 0 & 0 & cc\\
       0 & 0 & 0 & -ss & -cc & 0\\
       cs & -ss & cc & 0 & 0 & -sc\\
       sc & cc & -ss & 0 & 0 & cs\\
       0 & 0 & 0 & sc & -cs & 0\\
     \end{pmatrix*}
     \begin{pmatrix}
     \dot{\rho}\\
     \rho \dot{\Theta}\\
     \rho \dot{\Phi}\\
     \rho \omega_x\\
     \rho \omega_y\\
     \rho \omega_z\\
     \end{pmatrix}\\
\end{align}
In matrix notation:

\begin{equation}
	\dot{\mathbf{q}}' = \hat{A}\dot{\mathbf{q}},
\end{equation} 
where     
     
\begin{equation}
\dot{\mathbf{q}}' =
\begin{pmatrix}
\dot{r}'_x \\
\dot{r}'_y \\
\dot{r}'_z \\
\dot{R}'_x \\
\dot{R}'_y \\
\dot{R}'_z\\
\end{pmatrix},
\dot{\mathbf{q}} =
\begin{pmatrix}
\dot{\rho}\\
\dot{\Theta}\\
\dot{\Phi}\\
\omega_x\\
\omega_y\\
\omega_z\\
\end{pmatrix} \notag \\
\end{equation}

     
The arclength is given by

\begin{equation}
s = \int_{a}^{b} \| \dot{\mathbf{q}}' \| dt = \int_{a}^{b} \sqrt{ \dot{\mathbf{q}}'^{T} \dot{\mathbf{q}}'}\\ dt
\end{equation}

and

\begin{equation}
(ds)^2 = (d\mathbf{q}')^{T}(d\mathbf{q}') = d\mathbf{q}^{T} \hat{A}^{T} \hat{A}d\mathbf{q} = d\mathbf{q}^{T} \mathbf{g} d\mathbf{q},
\end{equation}
where $\mathbf{g}$ is the metric tensor

\begin{equation}
\mathbf{g}=
\begin{pmatrix}
\mathbf{G} & \mathbf{C}\\
\mathbf{C}^T & \mathbf{K}
\end{pmatrix}
\end{equation}
where the submatrices $\mathbf{G}$, $\mathbf{K}$ and $\mathbf{C}$ are 

\begin{align}
\mathbf{G} &=
\begin{pmatrix}
1 & 0      & 0\\
0 & \rho^2 & 0\\
0 & 0      & \rho^2
\end{pmatrix}\\
\mathbf{K} &=
\rho^2
\begin{pmatrix}
\sin^2\Theta & 0            & 0\\
0            & \cos^2\Theta & 0\\
0            & 0            & 1
\end{pmatrix}\\
\mathbf{C} &=
-\rho^2\sin^2(2\Theta)
\begin{pmatrix}
0 & 0 & 0\\
0 & 0 & 0\\
0 & 0 & 1
\end{pmatrix}
\end{align}
the inverse of the metric tensor $\mathbf{g}^{-1}$
\begin{equation}
\mathbf{g}^{-1}=
\begin{pmatrix}
\mathbf{V} & \mathbf{W}\\
\mathbf{W}^T & \mathbf{U}
\end{pmatrix}
\end{equation}
where the submatrices $\mathbf{V}$, $\mathbf{W}$ and $\mathbf{U}$ are 

\begin{align}
\mathbf{V} &=
\begin{pmatrix}
1 & 0      & 0\\
0 & 1/\rho^2 & 0\\
0 & 0      & 1/\rho^2\cos^2(2\Theta)
\end{pmatrix}\\
\mathbf{U} &=
\frac{1}{\rho^2}
\begin{pmatrix}
1/\sin^2\Theta & 0            & 0\\
0            & 1/\cos^2\Theta & 0\\
0            & 0            & 1/\cos^2(2\Theta)
\end{pmatrix}\\
\mathbf{W} &=
\frac{\sin(2\Theta)}{\rho^2\cos^2(2\Theta)}
\begin{pmatrix}
0 & 0 & 0\\
0 & 0 & 0\\
0 & 0 & 1
\end{pmatrix}
\end{align}

The determinant of the metric tensor is given by

\begin{align}
g &=
\mid\mathbf{g}\mid=
\begin{vmatrix}
\mathbf{G} & \mathbf{C}\\
\mathbf{C}^T & \mathbf{K}
\end{vmatrix}
=
\begin{vmatrix}
\mathbf{G} & \mathbf{C}\\
0 & \mathbf{K} - \mathbf{C}^T \mathbf{G}^{-1} \mathbf{C}
\end{vmatrix}\\
  &=
\mid \mathbf{G} \mid \cdot \mid\mathbf{K} - \mathbf{C}^T \mathbf{G}^{-1} \mathbf{C} \mid
=
\frac{\rho^{10}}{16}\sin^2(4\Theta)
\end{align}

\begin{equation}
	\sqrt{g}=\frac{\rho^5}{4}\sin(4\Theta)
\end{equation}

The kinetic energy operator of a particle with mass $\mu$ in a curvilinear coordinate system of $N$ dimensions is given by

\begin{equation}
	T = -\frac{\hbar^2}{2\mu} \sum_{i=1}^{N} \sum_{j=1}^{N} \frac{1}{\sqrt{g}} \frac{\partial}{\partial q_i} \Big(\sqrt{g} g^{ij} \frac{\partial}{\partial q_j}\Big)
\end{equation}
where $g^{ij}$ is the inverse metric tensor. The momentum vector is given by

\begin{equation}
	\mathbf{p} = i\hbar 
	\begin{pmatrix}
		\partial/\partial q_1\\
		\vdots\\
		\partial/\partial q_N\\
	\end{pmatrix}
\end{equation}

With $\mathbf{\omega}$ expressed in Euler angles 

\begin{equation}
\mathbf{\omega} = 
\begin{pmatrix}
	d\Omega_x\\
	d\Omega_y\\
	d\Omega_z
\end{pmatrix}
\end{equation}

we get the kinetic energy

\begin{align*}
	-\frac{1}{\hbar^2}\hat{T} &= -\frac{1}{2\mu \rho^5 \sin(4\Theta)} \mathbf{p}^T \Big(\rho^5 \sin(4\Theta) \mathbf{g}^{-1}\Big) \mathbf{p}\\
	  &= \frac{1}{2\mu \rho^5 \sin(4\Theta)}\Bigg[ \frac{\partial}{\partial \rho} \Bigg(\rho^5 \sin(4 \Theta) \frac{\partial}{\partial \rho}\Bigg) + \frac{\partial}{\partial \Theta}\Bigg(\rho^3\sin(4\Theta)\frac{\partial}{\partial\Theta}\Bigg)\\ &+\frac{\partial}{\partial\Phi}\Bigg(2\rho^3\tan(2\Theta)\frac{\partial}{\partial\Phi} + 2\tan^2(2\Theta)\cos(2\Theta)\frac{\partial}{\partial\Omega_z}\Bigg)\\
	  &+\frac{\partial}{\partial\Omega_x}\Bigg(4\rho^3\cot(\Theta)\cos(2\Theta)\frac{\partial}{\partial\Omega_x}\Bigg)\\
	  &+\frac{\partial}{\partial\Omega_y}\Bigg(4\rho^3\tan(2\Theta)\cos(2\Theta)\frac{\partial}{\partial\Omega_y}\Bigg)\\
	  &+\frac{\partial}{\partial\Omega_z}\Bigg(2\rho^3\tan^2(2\Theta)\frac{\partial}{\partial\Phi}+2\rho^3\tan(2\Theta)\frac{\partial}{\partial\Omega_z}\Bigg)\Bigg]\\
	  &=\frac{1}{2\mu}\Bigg[\frac{1}{\rho^5}\frac{\partial}{\partial\rho}\Bigg(\rho^5\frac{\partial}{\partial\rho}\Bigg) + \frac{1}{\rho^2\sin(4\Theta)}\frac{\partial}{\partial\Theta}\Bigg(\sin(4\Theta)\frac{\partial}{\partial\Theta}\Bigg)\\
	  &+\frac{1}{\rho^2\cos^2(2\Theta)}\frac{\partial^2}{\partial\Phi^2} + \frac{\sin(2\Theta)}{\rho^2\cos^2(2\Theta)}\frac{\partial}{\partial\Phi}\frac{\partial}{\partial\Omega_z}\\
	  &+\frac{1}{\rho^2\sin^2(\Theta)}\frac{\partial^2}{\partial\Omega^2_x} + \frac{1}{\rho^2\cos^2(\Theta)}\frac{\partial^2}{\partial\Omega^2_y} + \frac{\sin(2\Theta)}{\rho^2\cos^2(2\Theta)}\frac{\partial}{\partial\Omega_z}\frac{\partial}{\partial\Phi}\\
	  &+ \frac{1}{\rho^2\cos^2(2\Theta)}\frac{\partial^2}{\partial\Omega^2_z}\Bigg]\\
	  &=\frac{1}{2\mu\rho^5}\frac{\partial}{\partial\rho}\Bigg(\rho^5\frac{\partial}{\partial\rho}\Bigg) + \frac{1}{2\mu\rho^2}\Bigg[\frac{1}{\sin(4\Theta)}\frac{\partial}{\partial\Theta}\Bigg(\sin(4\Theta)\frac{\partial}{\partial\Theta}\Bigg)\\
	  &+\frac{1}{\cos^2(2\Theta)}\frac{\partial^2}{\partial\Phi^2} \Bigg]+\frac{1}{2\mu\rho^2}\Bigg[\frac{1}{\sin^2(\Theta)}\frac{\partial^2}{\partial\Omega^2_x} + \frac{1}{\cos^2(\Theta)}\frac{\partial^2}{\partial\Omega^2_y} + \frac{1}{\cos^2(2\Theta)}\frac{\partial^2}{\partial\Omega^2_z}\\
	  &+ \frac{2\sin(2\Theta)}{\cos^2(2\Theta)}\frac{\partial}{\partial\Phi}\frac{\partial}{\partial\Omega_z}\Bigg].
\end{align*}

The volume element in a general coordinate system of  N dimensions is given by

\begin{equation}
d^Nv = g^{1/2}\prod_{i=1}^{N} dq_i
\end{equation}
The Euler angles are given by

\begin{equation}
\begin{pmatrix}
d\Omega_x\\
d\Omega_y\\
d\Omega_z
\end{pmatrix}
=
\mathbf{A}\begin{pmatrix}
d\alpha\\
d\beta\\
d\gamma
\end{pmatrix}
\end{equation}
where

\begin{equation}
\mathbf{A}=
	\begin{pmatrix}
		-\sin\beta\cos\gamma & \sin\gamma & 0\\
		\sin\beta\sin\gamma  & \cos\gamma & 0\\
		\cos\beta 			 & 0          & 1
	\end{pmatrix}
\end{equation}

\begin{equation}
d\mathbf{q} =
\begin{pmatrix}
d\rho\\
d\Theta\\
d\Phi\\
d\Omega_x\\
d\Omega_y\\
d\Omega_z\\
\end{pmatrix},
d\mathbf{q}' =
\begin{pmatrix}
d\rho\\
d\Theta\\
d\Phi\\
d\alpha\\
d\beta\\
d\gamma\\
\end{pmatrix} \notag \\
\end{equation}

\begin{equation}
	ds^2 = (d\mathbf{q})^T\mathbf{g}d\mathbf{q} = (d\mathbf{q}')^T\mathbf{g}'d\mathbf{q}'
\end{equation}
with

\begin{equation}
	\mathbf{B} = 
	\begin{pmatrix}
		\mathbf{I} & 0\\
		0      & \mathbf{A}
	\end{pmatrix}
\end{equation}

\begin{equation}
	\mathbf{g}'=\mathbf{B}^T\mathbf{g}\mathbf{B}
\end{equation}
and the determinant is then

\begin{equation}
g'^{1/2}=(\mid\mathbf{B}\mid^2 \mid\mathbf{g}\mid)^{1/2} = (\mid\mathbf{A}\mid^2 \mid\mathbf{g}\mid)^{1/2}
  = \frac{\rho^5}{4}\sin(4\Theta)\sin\beta
\end{equation}

In our set up we thus get 

\begin{equation}
d^6v = g^{1/2}\prod_{i=1}^{6} dq_i = \frac{\rho^5}{4}\sin(4\Theta)\sin\beta d\rho d\Theta d\Phi d\alpha d\beta d\gamma
\end{equation}

Now, we make a transformation of the angles (Kuppermann)

\begin{align*}
	\Theta &= \pi/4-\theta/2\\
	\Phi &= \pi/4-\phi/2
\end{align*}

and

\begin{align*}
	\frac{\partial}{\partial\Theta} &= -2\frac{\partial}{\partial\theta}\\ 						\frac{\partial}{\partial\Phi} &= -2\frac{\partial}{\partial\phi}\\ 		
\end{align*}

\begin{align*}
	&\sin(4\Theta)   =\sin(2\theta)\\
	&\cos^2(2\Theta) =\sin^2(\theta)\\
	&\sin^2(2\Theta) =\cos^2(\theta)\\
	&\cos^2\Theta    =\frac{1}{2}(1+\sin\theta)\\
	&\sin^2\Theta    =\frac{1}{2}(1-\sin\theta)\\
\end{align*}
The corresponding volume element is then

\begin{equation}
	d^6 v = \frac{1}{8}\rho^5 \sin\theta\cos\theta\sin\beta d\rho d\theta d\phi d\alpha d\beta d\gamma
\end{equation}

Now with

\begin{equation}
	\mathbf{P} = 
	-i\hbar
	\begin{pmatrix}
	\partial/\partial\rho\\
	\partial/\partial\theta\\
	\partial/\partial\phi
	\end{pmatrix}
	=
	\begin{pmatrix}
	P_{\rho}\\
	P_{\theta}\\
	P_{\phi}
	\end{pmatrix}
\end{equation}
and

\begin{equation}
	\mathbf{J} = 
	-i\hbar
	\begin{pmatrix}
	\partial/\partial\Omega_x\\
	\partial/\partial\Omega_y\\
	\partial/\partial\Omega_z
	\end{pmatrix}
	=
	\begin{pmatrix}
	J_x\\
	J_y\\
	J_z
	\end{pmatrix}
\end{equation}

The kinetic energy operator then becomes

\begin{align*}
	\hat{T} &=-\frac{\hbar^2}{2\mu}\Bigg[\frac{1}{\rho^5}\frac{\partial}{\partial\rho}\rho^5\frac{\partial}{\partial\rho} + \frac{4}{\rho^2}\Bigg(\frac{1}{\sin(2\theta)}\frac{\partial}{\partial\theta}\sin(2\theta)\frac{\partial}{\partial\theta}\\
	&+\frac{1}{\sin^2(\theta)}\frac{\partial^2}{\partial\phi^2} \Bigg)\Bigg]-\frac{1}{\mu\rho^2}\Bigg[\frac{J^2_x}{(1-\sin\theta)} + \frac{J^2_y}{(1+\sin\theta)} +\frac{J^2_z}{2\sin^2\theta}\Bigg]\\
	&+ \frac{4i\hbar\cos\theta J_z}{2\mu\rho^2\sin^2\theta}\frac{\partial}{\partial\phi}
\end{align*}

If we only consider $J=0$ states, the Hamiltonian reduces to 

\begin{equation}
H_{0} = -\frac{\hbar^{2}}{2 \mu} \Bigg[ \frac{1}{\rho^{5}} \frac{\partial}{\partial \rho} \rho^{5} \frac{\partial}{\partial \rho} + \frac{4}{\rho^{2}}\Big( \frac{1}{\sin(2\theta)} \frac{\partial}{\partial \theta} \sin(2\theta) \frac{\partial}{\partial \theta} + \frac{1}{\sin^{2}(\theta)} \frac{\partial^{2}}{\partial \phi^{2}} \Big) \Bigg] + V(\rho, \theta, \phi)
\end{equation}

To simplify the kinetic terms in the variables rho and theta, make a transform of the wave function such that

\begin{equation}
\psi = \rho^{5/2} \cos^{1/2}(\theta) \Psi
\end{equation}

The Schr{\"o}dinger equation is then

\begin{equation}
H \psi = E \psi 
\end{equation}

Where the transformed Hamiltonian operator is given by

\begin{equation}
H = \rho^{5/2} \cos^{1/2}{(\theta)} H_{0} \rho^{-5/2} \cos^{-1/2}{(\theta)}
\end{equation}

with

\begin{equation}
\rho^{5/2} \cos^{1/2}{(\theta)} \frac{1}{\rho^{5}} \frac{\partial}{\partial \rho} \rho^{5} \frac{\partial}{\partial \rho} \Big( \rho^{-5/2} \cos^{-1/2}{(\theta)} \Big) = -\frac{15}{4} \frac{1}{\rho^{2}} + \frac{\partial^{2}}{\partial \rho^{2}}
\end{equation}

and

\begin{align}
&\frac{4}{\rho^{2}}\cos^{1/2}\theta \Big( \frac{1}{\sin(2\theta)} \frac{\partial}{\partial \theta} \sin(2\theta) \frac{\partial}{\partial \theta} + \frac{1}{\sin^{2}(\theta)} \frac{\partial^{2}}{\partial \phi^{2}} \Big) \cos^{-1/2}\theta\\
&=\frac{4}{\rho^{2}}\Big( \frac{1}{\sin\theta} \frac{\partial}{\partial \theta} \sin\theta \frac{\partial}{\partial \theta} + \frac{1}{\sin^{2}\theta} \frac{\partial^{2}}{\partial \phi^{2}} + 1 + \frac{1}{4}\tan^2\theta \Big)
\end{align}

\begin{align}
-\frac{2\mu}{\hbar^2}\mathbf{H} &= -\frac{15}{4} \frac{1}{\rho^{2}} + \frac{\partial^{2}}{\partial \rho^{2}}\\
& + \frac{4}{\rho^{2}}\Big( \frac{1}{\sin\theta} \frac{\partial}{\partial \theta} \sin\theta \frac{\partial}{\partial \theta} + \frac{1}{\sin^{2}\theta} \frac{\partial^{2}}{\partial \phi^{2}} + 1 + \frac{1}{4}\tan^2\theta \Big)\\
&= \frac{1}{4\rho^{2}}(1 + 4\tan^2\theta) + \frac{\partial^{2}}{\partial \rho^{2}}\\
& + \frac{4}{\rho^{2}}\Big( \frac{1}{\sin\theta} \frac{\partial}{\partial \theta} \sin\theta \frac{\partial}{\partial \theta} + \frac{1}{\sin^{2}\theta} \frac{\partial^{2}}{\partial \phi^{2}} \Big)\\
&= \frac{1}{\rho^{2}}\Big(\frac{1}{\cos^2\theta} - \frac{3}{4}\Big) + \frac{\partial^{2}}{\partial \rho^{2}}\\
& + \frac{4}{\rho^{2}}\Big( \frac{1}{\sin\theta} \frac{\partial}{\partial \theta} \sin\theta \frac{\partial}{\partial \theta} + \frac{1}{\sin^{2}\theta} \frac{\partial^{2}}{\partial \phi^{2}} \Big)\\
\end{align}

\subsubsection{Symmetries}

The fact that each permutation only causes a
simple translation or translation plus inversion of w is a key
strength of these democratic angular coordinates.(Blume 02). Note that
the restriction to L50 states, the closed-shell nature of electronic
degrees of freedom, and the treatment of bosons with
nuclear spin I50 only ~see below! leads to vibrational wave
functions of A1 symmetry.4


\subsection{Hyperradial equations}

The hyperradius is given by

\begin{equation}
\rho = (R + r)^{1/2}
\end{equation}

In hyperspherical coordinates the three-body Schr{\"o}diner equation becomes

\begin{equation}
\Bigg(-\frac{1}{2 \mu_{3}}\frac{\partial^2}{ \partial \rho^2} + \frac{\Lambda^2 + 15/4}{2 \mu_{3} \rho^2} + V \Bigg)\psi(\rho,\Omega) = E\psi(\rho,\Omega)
\end{equation}

\begin{equation}
\psi = \sum \frac{f{\rho}}{\rho^{5/2}} \frac{\phi{(\rho,\Omega)}}{\sin(\alpha) \cos(\alpha)}
\end{equation}

The Hamiltonian

\begin{equation}
H(\rho,\Omega) = T(\rho) + H_{ad}(\rho,\Omega)
\end{equation}

\begin{equation}
H_{ad}(\rho,\Omega) = \frac{\Lambda^2}{2 \mu_{3} \rho^2} + \sum_{i<j} V(r_{ij})
\end{equation}

\begin{equation}
H_{ad} \Phi_{\nu}{(\rho ; \Omega)}= U_{\nu}{(\rho)} \Phi_{\nu}{(\rho ; \Omega)}
\end{equation}


Smith-Whitten (democratic) hyperangles

\begin{equation}
r_{ij} = \frac{d_{ij} \rho}{\sqrt{2}} \Big[1+\sin(\theta) \cos{(\varphi + \varphi_{ij})}\Big]^{1/2}
\end{equation}

\begin{align*}
\varphi_{12} &= 2 \arctan{(m_{3} / \mu_{3})} \\
\varphi_{23} &= 0 \\
\varphi_{31} &= -2 \arctan{(m_{2} / \mu_{3})}
\end{align*}

\begin{equation}
d_{ij}^{2} =  \frac{m_{k}}{\mu_{3}} \frac{m_{i} + m_{j}}{m_{i} + m_{j} + m_{k}}
\end{equation}

\begin{equation}
\Lambda^2 = T_{\theta} + T_{\phi} + T_{rot}  
\end{equation}

\begin{equation}
T_{\theta} = -\frac{4}{\sin(2\theta)} \frac{\partial}{\partial \theta} \sin(2\theta) \frac{\partial}{\partial \theta}
\end{equation}

\begin{equation}
T_{\phi} = \frac{2}{\sin^2{(2 \theta)}} \Big( i \frac{\partial}{\partial \phi} - \frac{\cos{\theta}}{2} J_{z}\Big)^2  
\end{equation}

\begin{equation}
T_{rot} = \frac{2}{1-\sin(\theta)} J_{x}^{2} + \frac{2}{1+\sin(\theta)} J_{y}^{2} + J_{z}^{2}  
\end{equation}

\subsection{Three identical particles}
\subsubsection{Permutation symmetries in Smith-Whitten coordinates}

\subsection{Effective three-body interaction}
The two-body interaction 

\begin{equation}
V(r) = d\cosh{(r/r_0)}
\end{equation}







\end{document}