 \documentclass{standalone}
 \usepackage{tikz}
 \usepackage{tikz-3dplot}
 \usetikzlibrary{calc}
 
 \begin{document}
 
  %Angle Definitions
%-----------------

%set the plot display orientation
%synatax: \tdplotsetdisplay{\theta_d}{\phi_d}
\tdplotsetmaincoords{60}{110}

%define polar coordinates for some vector
%TODO: look into using 3d spherical coordinate system
\pgfmathsetmacro{\rvec}{.8}
\pgfmathsetmacro{\thetavec}{30}
\pgfmathsetmacro{\phivec}{80}

\pgfmathsetmacro{\rveca}{.8}
\pgfmathsetmacro{\thetaveca}{50}
\pgfmathsetmacro{\phiveca}{320}

\pgfmathsetmacro{\rvecb}{.8}
\pgfmathsetmacro{\thetavecb}{50}
\pgfmathsetmacro{\phivecb}{40}


%start tikz picture, and use the tdplot_main_coords style to implement the display 
%coordinate transformation provided by 3dplot
\begin{tikzpicture}[scale=5,tdplot_main_coords]

%set up some coordinates 
%-----------------------
\coordinate (O) at (0,0,0);

%determine a coordinate (P) using (r,\theta,\phi) coordinates.  This command
%also determines (Pxy), (Pxz), and (Pyz): the xy-, xz-, and yz-projections
%of the point (P).
%syntax: \tdplotsetcoord{Coordinate name without parentheses}{r}{\theta}{\phi}
\tdplotsetcoord{P}{\rvec}{\thetavec}{\phivec}
\tdplotsetcoord{F}{\rveca}{\thetaveca}{\phiveca}
\tdplotsetcoord{R}{\rvecb}{\thetavecb}{\phivecb}
\coordinate (M) at ($ (P) !.5! (R) $);


%draw figure contents
%--------------------

%draw the main coordinate system axes
\draw[thick,->] (0,0,0) -- (1,0,0) node[anchor=north east]{$x$};
\draw[thick,->] (0,0,0) -- (0,1,0) node[anchor=north west]{$y$};
\draw[thick,->] (0,0,0) -- (0,0,1) node[anchor=south]{$z$};
%draw a vector from origin to point (P) 
\draw[-stealth,color=red] (O) -- (P) node[pos=0.5, right]{$x_{i}$};
\draw[-stealth,color=red] (O) -- (F) node[pos=0.5, right]{$x_{k}$};
\draw[-stealth,color=red] (O) -- (R) node[pos=0.5, right]{$x_{j}$};

\draw[thick,->,color=blue] (P) -- (R) node[pos=0.5, right]{$x_{ij}$};
\draw[thick,->,color=blue] (M) -- (F) node[pos=0.5, below]{$x_{ij,k}$};
%\draw[-stealth,color=red] (O) -- (R);

%draw projection on xy plane, and a connecting line
\draw[dashed, color=red] (O) -- (Pxy);
\draw[dashed, color=red] (P) -- (Pxy);

\draw[dashed, color=red] (O) -- (Fxy);
\draw[dashed, color=red] (F) -- (Fxy);

\draw[dashed, color=red] (O) -- (Rxy);
\draw[dashed, color=red] (R) -- (Rxy);

%draw the angle \phi, and label it
%syntax: \tdplotdrawarc[coordinate frame, draw options]{center point}{r}{angle}{label options}{label}
\tdplotdrawarc{(O)}{0.8}{0}{\phivec}{anchor=north}{$\phi_i$}
\tdplotdrawarc{(O)}{0.2}{0}{\phiveca}{anchor=north}{$\phi_j$}
\tdplotdrawarc{(O)}{0.4}{0}{\phivecb}{anchor=north}{$\phi_k$}


%set the rotated coordinate system so the x'-y' plane lies within the
%"theta plane" of the main coordinate system
%syntax: \tdplotsetthetaplanecoords{\phi}
\tdplotsetthetaplanecoords{\phivec}

%draw theta arc and label, using rotated coordinate system
\tdplotdrawarc[tdplot_rotated_coords]{(0,0,0)}{0.8}{0}{\thetavec}{anchor=south west}{$\theta_i$}

%draw some dashed arcs, demonstrating direct arc drawing
\draw[dashed,tdplot_rotated_coords] (\rvec,0,0) arc (0:90:\rvec);
\draw[dashed] (\rvec,0,0) arc (0:90:\rvec);


\tdplotsetthetaplanecoords{\phiveca}
\tdplotdrawarc[tdplot_rotated_coords]{(0,0,0)}{0.8}{0}{\thetaveca}{anchor=south east}{$\theta_k$}


%draw some dashed arcs, demonstrating direct arc drawing
\draw[dashed,tdplot_rotated_coords] (\rveca,0,0) arc (0:90:\rveca);
\draw[dashed] (\rveca,0,0) arc (0:\phiveca:\rveca);

\tdplotsetthetaplanecoords{\phivecb}
\tdplotdrawarc[tdplot_rotated_coords]{(0,0,0)}{0.8}{0}{\thetavecb}{anchor=south east}{$\theta_j$}


%draw some dashed arcs, demonstrating direct arc drawing
\draw[dashed,tdplot_rotated_coords] (\rvecb,0,0) arc (0:90:\rvecb);
\draw[dashed] (\rvecb,0,0) arc (0:90:\rvecb);


\end{tikzpicture}
\end{document}