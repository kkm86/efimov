\chapter{Results and Discussion}
The results presented here concern the effective three-body potential curves $U_{\nu}$. Signatures of the Efimov effect can be revealed in the structure of these adiabatic hyperspherical potentials, with the Efimov channel corresponding to the one emergent attractive potential converging to \eqref{eq:efimov_channel} in the region $r_0 \ll \rho \ll
\abs{a}$.

The adiabatic potential curves were obtained by numerically solving  \eqref{adiabatic} using the method described in \cref{chapter:5}. The setup described in this work was used to accurately reproduce the results presented by Blume in \cite{Blume2002}, which we show in \cref{fig:res_2}. The adiabatic potential curves were here numerically calculated with a potential strong enough to support two $s$-wave bound states and the potential depth $d$ was tuned to yeild a scattering length of $a=228$ a.u. The adiabatic potential curves $U_{\nu}$ with $\nu = 0-7$ are plotted as functions of the hyperradius $\rho/a$. In this figure we can see that the two lowest potential curves $\nu = 0$ and $\nu = 1$ are converging to the two-body $s$-wave binding energies, which represent channels with one two-body bound  state and a free particle. Here the channel index $\nu=1$ corresponds to the Efimov channel. The higher lying channels $\nu>1$ are continuum channels and the channel index $\nu=0$ represents a deeply bound state.

\begin{figure}
	\includegraphics[width=\linewidth]{adiabatic.pdf}
	\caption{The adiabatic potential curves $U_{\nu}$ for $\nu=0-7$ are plotted as functions of the hyperradius $\rho$ for $a=228$ a.u. The index labelled $\nu=1$ corresponds to the Efimov channel.}
	\label{fig:res_2}
\end{figure}

The results presented in \cref{fig:res_3,fig:res_4,fig:res_5,fig:res_6} were, in the cases where $a<0$, obtained by using values of $d$ that approached pole $\mathrm{I}$ from the right. For these values of $d$ the potential is too weak to support a two-body bound state. Similarly, for the results obtained for $a>0$ we used values of $d$ that approached pole $\mathrm{I}$ from the left, thus supporting a single two-body bound state. 

If \eqref{eq:efimov_channel} holds, then we expect the potential curves to converge towards \eqref{eq:efimov_channel} for $\abs{a} \rightarrow \infty$. This behaviour should be evident if the potentials are multiplied by $2 \mu \rho^2$ and plotted as 

\begin{equation}\label{eq:lambda}
\xi(\rho) = 2 \mu \rho^2 U_{\nu}(\rho) + \frac{1}{4},
\end{equation}
since these curves should approach the universal value $-s_0^2 (\simeq -1.0125$ for $J=0^+$ states).

In \cref{fig:res_3} the potential curves asymptotically associated with the lowest continuum channel for $a<0$ are plotted as functions of the hyperradius $\rho$ for four different values of $a$. As $\rho \rightarrow \infty$, the curves approach the eigenvalues of the kinetic energy, which correspond to $\lambda(\lambda + 4) + 4$ with $\lambda = 0$. In the intermediate region, the two effective potentials for the largest $\abs{a}$ seem to converge to $-s_0^2$, which is the straight dashed line in \cref{fig:res_3}. 

\begin{figure}
	\includegraphics[width=\linewidth]{plotneg_dashed.pdf}
	\caption{Three-body effective potentials for $a<0$. The horizontal dashed line is the value $-s_0^2$, which the Efimov potential takes on for $\rho\gg r_0$.}
	\label{fig:res_3}
\end{figure}
Convergence of the curve for $a \rightarrow -\infty$ was checked by computing this potential for two different numbers of mesh points $N_{\theta}=N_{\phi}=80$ and $100$, see \cref{fig:res_4}. The potential curve obtained with a larger number of mesh points can be seen to converge to $-s_0^2$ in the intermediate region at hyperradii up to approximately $\rho=8000$ a.u. However, we were not able to obtain convergence for $\rho>1 \times 10^4$ a.u. 

\begin{figure}
	\includegraphics[width=\linewidth]{diffdiff.pdf}
	\caption{Three-body effective potentials for $a=-2702020$ a.u. obtained using $N_{\theta}=N_{\phi}=80$ and $N_{\theta}=N_{\phi}=100$ mesh points. The potential obtained with $N_{\theta}=0$ converges to $-s_0^2$ for larger $\rho$.}
	\label{fig:res_4}
\end{figure}
In \cref{table:Res_1} we can identify this approximative convergence at $\rho=8000$ a.u. for the potential obtained with $N_{\theta}=100$ mesh points, whereas the potential obtained with $N_{\theta}=80$ mesh points has started to converge to the kinetic energy at this hyperradius.  

\begin{table}[h!]
	\centering
	\footnotesize
	\begin{adjustwidth}{-0.1cm}{}
		\tabcolsep=0.10cm
		\begin{tabular}{||c c c c c c c||} 
			\hline
			$a$ (a.u.) & $N_{\theta}$ & $\xi(\rho = 10 $ a.u.) & $\xi(100 $ a.u.) & $\xi(1000 $ a.u.) & $\xi(5000 $ a.u.) & $\xi(10000 $ a.u.)  \Tstrut\Bstrut \\ [0.7ex]
			\hline\hline
			$a_1$   & 60  & 3.77527815 & $-$2.61220428 & $-$1.14442715 & $-$1.03421746 & $-$1.01917833 \\
			$a_1$   & 80  & 3.77527815 & $-$2.61220428 & $-$1.14442715 & $-$1.03421174 & $-$1.01917965 \\
			$a_1$   & 100  & 3.77527867 & $-$2.61220031 & $-$1.14442721 & $-$1.03421737 & $-$1.01918011 \\
			$a_2$   & 60  & 3.77526186 & $-$2.61268577 & $-$1.14585781 & $-$1.04076707 & $-$1.03213893 \\
			$a_2$   & 80  & 3.77526185 & $-$2.61268577 & $-$1.14585781 & $-$1.04062190 & $-$1.03214034 \\
			$a_2$   & 100  & 3.77526185 & $-$2.61268509 & $-$1.14585820 & $-$1.04072842 & $-$1.03214025 \\ [1ex] 
			\hline
			\hline
			$a$ (a.u.) & $N_{\theta}$ & $\xi(14900 $ a.u.) & $\xi(15000 $ a.u.) & $\xi(16000 $ a.u.) & $\xi(18000 $ a.u.) & $\xi(20000 $ a.u.)  \Tstrut\Bstrut \\ [0.7ex]
			\hline\hline
			$a_1$   & 60  & $-$1.01252783 & $-$1.01240910 & $-$1.01137389 & $-$1.00945098 & $-$1.00769268   \\
			$a_1$   & 80 & $-$1.01253099 & $-$1.01242168 & $-$1.01137813 & $-$1.00945850 & $-$1.00770842 \\
			$a_1$   & 100  & $-$1.01253102 & $-$1.01242296 & $-$1.01137768 & $-$1.00945888 & $-$1.00770548 \\
			$a_2$   & 60 & $-$1.03177291 & $-$1.03178238 & $-$1.03203010 & $-$1.03267303 & $-$1.03348102  \\
			$a_2$   & 80  & $-$1.03177594 & $-$1.03179508 & $-$1.03203428 & $-$1.03268000 & $-$1.03349676 \\
			$a_2$   & 100 & $-$1.03175920 & $-$1.03179379 & $-$1.03203408 & $-$1.03268093 & $-$1.03349752  \\ [1ex] 
			\hline
		\end{tabular}
	\end{adjustwidth}
	\caption{Three-body potential values at different hyperradii for $a_1 = -2702020$ a.u. $a_1 = 1966590$ a.u. corrsponding to \cref{fig:res_4}.}
	\label{table:Res_2}
\end{table} 

In \cref{fig:res_4} the potential curves associated with the weakly bound dimer for $a>0$ are plotted as functions of the hyperradius $\rho$, for four different values of $a$. The effective potentials $U_{\nu}$ are expected to converge to the weakly bound channel \eqref{eq:weakdimer}, whose energy is approximately given by $-1/ma^2$. This phenomenon is demonstrated at
large $\rho$ in \cref{fig:res_5} by the $-\rho^2$ divergence of the potential curves. As the magnitude of the scattering length is increased, the effective potential should converge to the Efimov potential (indicated by the horizontal dashed line at $-s_0^2$ in  \cref{fig:res_5}) in the intermediate range $r_0 \ll \rho \ll |a|$. This behaviour can indeed be identified for the potential curve for the largest $a$ in \cref{fig:res_5}.

\begin{figure}
	\includegraphics[width=\linewidth]{plotpos.pdf}
	\caption{Three-body effective potentials for $a>0$. The horizontal dashed line is $-s_0^2$.}
	\label{fig:res_5}
\end{figure}

The numerically calculated potentials in the form \eqref{eq:lambda} for two different negative scattering lengths were compared with the analytically derived eigenvalues $\lambda_0(\rho)$ for the hyperangular Faddeev equation \eqref{eq:faddeev_hyperang}. These eigenvalues, obtained at different hyperradii, are an adiabatic potential which corresponds to the effective three-body potentials calculated in this work through \eqref{eq:faddeev_effectivepot}. The adiabatic potential $\lambda_0(\rho/\abs{a})$ was determined from the transcendental equation \eqref{eq:transcendental} using \textit{Mathematica}. The analytically derived adibatic potential is plotted together with the numerically calculated potentials \eqref{eq:lambda} as functions of $\rho/\abs{a}$ in \cref{fig:res_6,fig:res_7}. Convergence of the potential curves with respect to the B-splines in $\theta$ and $\phi$ was checked by computing them for different numbers of mesh points $N_{\theta}$ and $N_{\theta}$. The convergence of the potential curves for $a=-2385$ a.u. and $a=-8720$ a.u. with an increasing number of mesh points is shown in \cref{fig:res_6,fig:res_7}, respectively. 

\begin{figure}
	\includegraphics[width=\linewidth]{sn2385.pdf}
	\caption{The convergence toward the analytically derived eigenvalues $\lambda_0$ for $a=-2385$ a.u. is shown for several different numbers of mesh points $N_{\theta}=N_{\phi}$.}
	\label{fig:res_6}
\end{figure}

\begin{figure}
	\includegraphics[width=\linewidth]{sn8720.pdf}
	\caption{The convergence toward the analytically derived eigenvalues $\lambda_0$ for $a=-8720$ a.u. is shown for several different numbers of mesh points $N_{\theta}=N_{\phi}$.}
	\label{fig:res_7}
\end{figure}

