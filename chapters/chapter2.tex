\chapter{Experimental Evidence}
\section{Efimov Trimers in Atomic Systems}
Ultracold atomic clouds provided the first staging ground for exploring Efimov physics and related few-body phenomena due to the ability to control atom-atom interactions near a Feshbach resonance\footnote{The physical origin of a Feshbach resonance is the coupling of the incident collision state, i.e. the entrance channel, to a bound molecular state, i.e. the closed channel, when the kinetic energy of the colliding atoms approaches the energy of interatomic potential of the bound state \cite{Feshbach}.} by means of an external magnetic field. In these extremely dilute gases, with densities $n$, the probability for collisions can reach unity by tuning the $s$-wave scattering length to the unitary regime $n\abs{a^3}\gg1$. In experiments with trapped ultracold atomic and molecular gases of alkali atoms with tunable two-body interactions, the existence of Efimov trimers have been inferred from resonantly enhanced loss rates for both atomic three-body recombination processes, when the Efimov state couples to the triatomic threshold ($0<a$), and in atom-dimer relaxation processes ($a>0$). 
\paragraph{Alkali atomic gases} 
The first observations of an Efimov resonance was reported\footnote{The actual experiment was performed in 2002 but it took several years before it was realized that this resonantly enhanced loss rate was caused by the formation of an Efimov trimer.} in 2006 by the Innsbruck-group of Grimm et al. in experiments with a trapped ultracold gas of $\prescript{133}{}{\mathrm{Cs}}$. To create a Bose-Einstein condensate they used the magnetically tunable interaction properties of caesium atoms in their energetically lowest state to vary the $s$-wave scattering length by applying magnetic fields. However, they were troubled by atoms escaping the trap. Atom loss occurs in three atom collisions when two of the atoms form a tightly bound pair because the kinetic energy release due to the pair formation is high enough for both the pair and the free atom to escape the trap. When three free atoms collide and resonantly couples to an Efimov state a fast decay channel into these deeply bound dimer states plus a free atom is formed. The processes of pair formation and atoms escaping the trap are therefore resonantly enhanced by the presence of Efimov trimers and it was these kinds of resonantly enhanced loss rates reported in \cite{Grimm:2006} that bore the first evidence for Efimov quantum states.

Later experiments have strengthened the evidence of these elusive trimers and in 2014 observations of the first excited Efimov state in an ultracold sample of caesium atoms confirmed the theoretically predicted universal scaling properties of two succesive Efimov states \cite{Huang2014}.

Efimov resonances in homogenic gases composed of a variety of atomic species have been observed including $\prescript{85}{}{\mathrm{Rb}}$ \cite{Klauss2017}, $\prescript{39}{}{\mathrm{K}}$ \cite{Potassium}, $\prescript{7}{}{\mathrm{Li}}$ \cite{Lithium7} and in three component Fermi gases of $\prescript{6}{}{\mathrm{Li}}$ \cite{Williams2009}. Efimov physics have been extended to involve heteronuclear states and resonances have been reported in, for example, heterogenous mixtures of Li-Cs and Li-Rb \cite{LithiumRubidium,LithiumCeasium}.

\paragraph{Helium trimers}
Clusters of helium atoms $\prescript{4}{}{\mathrm{He}}_3$ were early on identified as a prime candidate for studying Efimov physics in a natural system because of the unique interaction properties of the diatomic helium potential \cite{Lim1977}. The inherent nonpolarizability of helium atoms due to their electron configuration causes an especially weak van der Waals attraction in the helium dimer $\prescript{4}{}{\mathrm{He}}_2$, a fact that is manifested by a very shallow potential with a naturally large positive $s$-wave scattering length supporting only one unusually large state with an extremely low binding energy \cite{Blume2019}. Theoretical calculations using the helium potential had predicted the existence of two consecutive Efimov states in the first and second excited state of the helium trimer \cite{Kamimura2012}.

The first experimental evidence of the existence of the Efimov state in the helium trimer was reported in 2015 \cite{Blume2015}. By using Coulomb explosion imaging of mass-selected clusters by means of matter wave diffraction, the group of Voigtsberger et al. was able to image the structure of $\prescript{4}{}{\mathrm{He}}_3$ and $\prescript{3}{}{\mathrm{He}}\prescript{4}{}{\mathrm{He}}_2$ clusters, which confirmed the Efimov characteristics of not only the first, but also the second excited $\prescript{4}{}{\mathrm{He}}$ trimer. 

\section{Efimov States in Nuclei }
The Efimov quantum effect was originally predicted to exist in the nuclei of tritium and in the Hoyle state\footnote{The Hoyle state is the second excited state in $\prescript{12}{}{\mathrm{C}}$, which is formed in two steps in the $3\alpha$ process where two $\alpha$-particles first react to form $\prescript{8}{}{\mathrm{Be}}$, which subsequently react with a third $\alpha$-particle to form $\prescript{12}{}{\mathrm{C}}$. This state is a resonance to the $\alpha + \prescript{8}{}{\mathrm{Be}}$ channel meaning its energy is close to the breakup threshold and it almost always decays back into its constituent parts of three $\alpha$-particles, however, very rarely they instead relax into the stable ground state configuration of $\prescript{12}{}{\mathrm{C}}$.} of $\prescript{12}{}{\mathrm{C}}$ because of the resonant character of nucleon-nucleon forces, which give rise to naturally large $s$-wave scattering lengths with nucleon-nucleon $s$-wave scattering lengths for the spin-triplet and the spin-singlet channel being $3$ and $15$ times the range of the nuclear forces \cite{Efimov:1970zz,Efimov:1971zz}.  However, these states have not yet been observed experimentally. 

\section{Four-body Recombination Connected to Efimov Trimers} 
The Efimov scenario is even richer. In connection to an Efimov trimer, a pair of four-body states can form when a fourth atom approaches. In accordance with the theoretical predictions, strong evidence for the existence of a pair of four-body states was provided in 2009 \cite{Grimm:2009}.