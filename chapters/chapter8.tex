\chapter{Conclusions and Outlook}\label{chap:8}
We have developed a code that solves the Schrödinger equation for three identical atoms, interacting with pair-wise potentials of, at least in principle, arbitrary form. Using this code and the scattering model described in \Cref{chapter:scattering_model} we have calculated three-body effective potentials for different scattering lengths. Our results have been compared to the analytical model describing the Efimov scenario. We have been able to get good agreement out to hyperradii of about $2 \times 10^4$ a.u. Further improvements in the selection and placement of the knots when using the B-spline basis for function approximation is needed to achieve convergence of the potential curves out to larger hyperradii. 

To speed up the code we are now working on replacing the current Lapack routine that computes the eigenvalues and eigenvectors of the generalized eigenvalue problem with the more efficient ARPACK package \cite{arpack}, which is designed to solve eigenvalue problems of large sparse matrices. 

In future studies we want to study different forms of the two-body potentials and their impact on the resonantly enhanced three-body recombination arising from the barrier in the adiabatic potential for negative scattering lengths. Of particular interest is model potentials that have a van der Waals tail, which is characteristic for the interaction between alkali atoms. This type of potentials has shown an unexpected universality in terms of the location of Efimov states \cite{3BP_origin}, but its implication for the resonantly enhanced three-body recombination has not been investigated. 

Furthermore, I want to explore the temperature dependence of both the Efimov states and the three-body parameter. Specifically, I want to develop a model for $\prescript{39}{}{\mathrm{K}}$, since the temperature dependency of the shape and position of an Efimov resonance found in $\prescript{39}{}{\mathrm{K}}$ in a recent experimental study lies in stark contrast to theoretical predictions \cite{Wacker_2018}. 