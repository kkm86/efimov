\documentclass[a4paper,11pt]{book}
\usepackage{import}
\usepackage{mypackages}

\usepackage{makeidx}
\makeindex
 
\begin{document}
 
\frontmatter

\begin{titlepage}
	\includepdf{Title}
\end{titlepage}

\chapter*{Abstract}
In contrast to the classical case, the quantum three-body problem is amenable to qualitative analysis and, in some cases, even to analytic solutions. In 1970, Vitaly Efimov predicted that resonant two-body forces could give rise to a series of bound energy levels in three-particle systems. When the short-ranged two-body forces approached resonance, he found a universal long-range three-body attraction emerging, giving rise to an infinite number of trimer states with binding energies obeying a discrete scaling law at resonance. This oddity in the three-body spectrum close to the zero-energy threshold has since become known as the quantum Efimov effect and the term Efimov physics now covers an array of universal phenomena arising in few-body systems, for particles interacting via short-ranged resonant interactions, whose appearance is due to an emergent three-body attractive force. 

In this thesis, I aim to summarize the theory of Efimov physics and the methodology used for developing a computer code that calculates the effective long-range three-body potentials, which give rise to the discrete Efimov energy spectrum. The calculations of these potentials were performed by formulating the problem in hyperspherical coordinates and introducing the adiabatic representation where the hyperradius is treated as an adiabatic parameter. 

\chapter*{Sammanfattning}
\begin{otherlanguage}{swedish} 
Till skillnad från det klassiska fallet så är det kvantmekaniska trekroppsproblemet mottagligt för kvalitativ analys och t.o.m. analytiskt lösbart i vissa fall. 1970 förutsade Vitaly Efimov att resonanta tvåkroppskrafter kunde ge upphov till en serie bundna energinivåer i trepartikelsystem. När de korträckviddiga tvåkroppskrafterna närmade sig resonans fann han att en universell långräckviddig trekroppsattraktion trädde fram, vilket gav upphov till ett oändligt antal trekroppstillstånd med bindningsenergier som följer en diskret skalningslag vid resonans. Denna märklighet i trekroppsspektret nära nollenergitröskeln har sedan dess kommit att kallas för den kvantmekaniska Efimoveffekten, och begreppet Efimovfysik täcker nu en rad universella fenomen som uppstår i fåkroppsproblem, för partiklar som växelverkar genom korträckviddiga resonanta växelverkningar, vars uppkomst beror på en framväxande attraktiv trekroppskraft.

I denna avhandling är mitt syfte att sammanfatta teorin om Efimovfysik och metodologin som har använts för att utveckla en datorkod som beräknar de effektiva långräckviddiga trekroppspotentialerna, som ger upphov till det diskreta Efimovenergispektret. Beräkningarna av dessa potentialer utfördes genom att formulera problemet i hypersfäriska koordinater och introducera den adiabatiska representationen där hyperradien behandlas som en adiabatisk parameter.
\end{otherlanguage}

%\chapter*{Dedication}
%Till Tristan

%\chapter*{Acknowledgements}
%I want to thank my supervisor Svante Jonsell.

\clearpage
\thispagestyle{empty}
 
\tableofcontents
\listoffigures
\listoftables
 
\mainmatter

\import{chapters/}{chapter1.tex}
\import{chapters/}{chapter2.tex}
\import{chapters/}{chapter3.tex}
\import{chapters/}{chapter4.tex}
\import{chapters/}{chapter5.tex}
\import{chapters/}{chapter6.tex}
\import{chapters/}{chapter7.tex}
\import{chapters/}{chapter8.tex}

\appendix
\import{appendices/}{appendix1new.tex}
\import{appendices/}{appendix3.tex}

 

 
\backmatter


\printbibliography

\end{document}